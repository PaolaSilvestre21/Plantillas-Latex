\documentclass[10pt]{article}  
\usepackage[utf8]{inputenc}
\usepackage{float}%Podemos usar el especificador [H] en las figuras para que se queden centradas
\usepackage{sidecap}% Para poner el texto de las imágenes al lado
\usepackage{anysize} 					% Para personalizar el ancho de  los márgenes
\marginsize{3cm}{3cm}{1cm}{1cm} % Izquierda, derecha, arriba, abajo
\usepackage{graphicx}
\usepackage{graphics}
\usepackage{inputenc}
\usepackage{float}
\usepackage{natbib}
\usepackage{enumitem}
\usepackage{lipsum}
\usepackage{transparent}
\usepackage{eso-pic}
%\usepackage{hyperref} 
% Para agregar encabezado y pie de página
\usepackage{fancyhdr} 
\pagestyle{fancy}
\fancyhf{}
\fancyhead[L]{\footnotesize USM- Sede Concepcion} %encabezado izquierda
\author{Nombre ,Apellido }
\date{Fecha/2023}
\fancyhead[R]{\footnotesize Departamento de Electrónica e Informática}   % dereecha
\fancyfoot[R]{\footnotesize P.S.C}  % Pie derecha
\fancyfoot[C]{\thepage}  % centro
\fancyfoot[L]{\footnotesize Ingenieria Informatica}  %izquierda
\renewcommand{\footrulewidth}{0.1pt}


\AddToShipoutPicture{
	\put(0,0){
		\parbox[b] [\paperheight]{\paperwidth}{
			\vfill
			\centering
			{\transparent{0.050}\includegraphics[scale=0.25]{/home/Latex/img/foto1.png}}
			\vfill
		}
	}
}



\begin{document}
	\begin{figure}
		\includegraphics[width=0.25\textwidth]{/home/Latex/img/foto1.png} %/home/user/img/
		\centering
	\end{figure}
	
	\title{Plantilla -USM}
	
	\maketitle
	
	\section{Pregunta A} 
	%\begin{enumerate}[label=(\alph*)]
	\begin{itemize}
		
		\lipsum
		
		%\end{enumerate}
		
		
	\end{itemize}
	\begin{itemize}
		\item A.2\\
		\lipsum
		
	\end{itemize}
	\begin{itemize}
		\item A.3
		
		\lipsum
	\end{itemize}
	\begin{itemize}
		\item A.4) \\
		
		\par Ejemplo 1:
		
		\begin{figure}[h] 
			
			\centering
			\includegraphics{/home/Latex/img/foto1.png} %/home/user/img/
			
		\end{figure}
		
	\end{itemize}
	
	\section{Pregunta B: }
	
	\begin{itemize}
		\lipsum
		
	\end{itemize}
	
	
	
	\begin{itemize}
		\lipsum
		
	\end{itemize}
	
	
	\section{Pregunta C}
	\begin{itemize}
		\lipsum
		
	\end{itemize}
	
	
	
	\begin{thebibliography}{2}
		
		
		\url{https://www.}\\
		\url{https://www.}
	\end{thebibliography}
	
	
\end{document}	